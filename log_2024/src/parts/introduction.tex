\section{Introduction} % (fold)
\label{sec:introduction}

  This paper proposes an alert mechanism for potential banking information theft on the user side. While much effort has been invested in developing safer E-Commerce and online transactional applications to protect vendors, more needs to be accomplished to protect the clients of such systems against phishing~\cite{nicholas:2024,stanley:2023}. Generally, they are presented with recommendations and good practices to guard against fraudsters attempting to steal critical banking information from them. We argue that fraudsters use ever-changing social engineering tactics which often render these practices obsolete. This paper proposes a proactive approach to trigger an alert on potential phishing of banking information.  A typical scenario of phishing tactics uses an email to lure the user into clicking on a link with a URL corresponding to a Web interface where the critical information is captured. Our approach analyses the link and look for similar, mainly phonetically-sounding,  domain names. We further compare the web pages and the forms that capture the information. We transform these forms into graphs and compare them using a convolutional graph neural network (GNN). The intuition behind our approach suggests that the co-existence of similar phonetically sounding domains with almost identical web forms raises the probability of some of these domains being \emph{illegitimate}. Thus, we trigger an alert cautioning the user about a potential information theft. Our approach focuses on banking information theft. We present the approach and report on the results of our experiments.

% section Introduction (end)
